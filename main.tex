\documentclass[12pt,letterpaper]{article}
% the package for network diagram
\usepackage{tikz-network}
% the package for the Axis plot
\usetikzlibrary{shapes}
\usepackage{pgfplots}
% the package for fill color in Axis Plot
\usepgfplotslibrary{fillbetween}
% Matlab code package
\usepackage[framed,numbered,autolinebreaks,useliterate]{mcode}




\usepackage[spanish]{babel} 
\usepackage[utf8]{inputenc}	
\spanishdecimal{.}
\usepackage{amsmath,amsthm,amsfonts,amssymb,amscd}
%Theorem Package
\newtheorem{theorem}{Theorem}
\usepackage{multirow,booktabs}
\usepackage[table]{xcolor}
\usepackage{fullpage}
\usepackage{lastpage}
\usepackage{enumitem}
\usepackage{fancyhdr}
\usepackage{mathrsfs}
\usepackage{wrapfig}
\usepackage{setspace}
\usepackage{calc}
\usepackage{multicol}
\usepackage{cancel}
\usepackage[retainorgcmds]{IEEEtrantools}
\usepackage[margin=3cm]{geometry}
\usepackage{floatrow}
\newlength{\tabcont}
\setlength{\parindent}{0.0in}
\setlength{\parskip}{0.05in}
\title{TALLER ALGEBRA LINEAL ISAAC RODRIGUEZ}
\newcommand\course{Soft Computing ENGG6570}	
\newcommand\semester{2021} 
\newcommand\yourname{Yaowen Mei}  
\theoremstyle{definition}
\newtheorem{defn}{Definición}
\newtheorem{reg}{Regla}
\newtheorem{ejer}{EJERCICIO}
\pagestyle{fancyplain}
\headheight 32pt
\lhead{\yourname\ \vspace{0.1cm} \\ \course}
\chead{\textbf{\Large Homework-1}}
\rhead{2021/05/28}


\begin{document}
\textbf{Question 1-1}.\textbf{Write the feedforward equations of h1, h2, h3 and y, as a function of the input
variables x1, x2 and x3. Note that the thresholds must be included:}
\begin{center}
\begin{tikzpicture}
% \Vertex[x=0,size=1,RGB,color={190,174,212},label=$x_0\equiv1$]{TI} 
\Vertex[x=3,size=1,label=$x_1$]{A} 
\Vertex[x=6,size=1,label=$x_2$]{B} 
\Vertex[x=9,size=1,label=$x_3$]{C}
% The orange Box
\Plane[x=-4.6,y=1.4,width=1.2,height=9.5,color=orange, NoBorder]
\Text[x=0,y=-4.0,fontsize=\SMALL,color=orange]{$f(x)=tanh(x)$}
% \Vertex[x=0,y=-4,size=1,RGB,color={190,174,212},label=$h_0\equiv1$]{TH}
\Vertex[x=2,y=-4,size=1,label=$h_1$]{D} 
\Vertex[x=6,y=-4,size=1,label=$h_2$]{E}
\Vertex[x=10,y=-4,size=1,label=$h_3$]{F}
% The Green Box
\Plane[x=-6.5,y=1.4,width=1.2,height=9.5,color=green!70!blue, NoBorder]
\Text[x=0,y=-6,fontsize=\SMALL,color=green!70!blue]{$g(x)=x$}
\Vertex[x=6,y=-6,size = 1,label=$y$]{Z}

\Edge[bend=-10,Direct=true,label=w1](A)(D)
\Edge[bend=-10,Direct=true,label=w2](A)(E)
\Edge[bend=10,Direct=true,label=w3](A)(F)

\Edge[bend=-10,Direct=true,label=w4](B)(E)
\Edge[bend=10,Direct=true,color=red,label=w5](B)(F)
\Edge[bend=-5,Direct=true,label=w6](C)(D)
\Edge[bend=5,Direct=true,label=w7](C)(F)

\Edge[bend=-5,Direct=true,color=red,label=v2](D)(Z)
\Edge[bend=1,Direct=true,label=v3](E)(Z)
\Edge[bend=5,Direct=true,label=v4](F)(Z)
\Edge[bend=-15,Direct=true,label=v1](A)(Z)
\end{tikzpicture}
\end{center}
\\
\textbf{Solution 1-1:} 
\\
From the diagram above, we can write:
\begin{equation}
    \boxed{{h_{ & 1}} = f\left( {{w_1}{x_1} + {w_6}{x_3} + \theta_{h1} } \right)}
\end{equation}
    
\begin{equation}
    \boxed{{h_{ & 2}} = f\left( {{w_2}{x_1} + {w_4}{x_2} + \theta_{h2} } \right)}
\end{equation}

\begin{equation}
    \boxed{{h_{ & 3}} = f\left( {{w_3}{x_1} + {w_5}{x_2} + {w_7}{x_3} + \theta_{h3} } \right)}
\end{equation}

\[y = g\left( {{v_1}x{ & _1} + {v_2}{h_1} + {v_3}{h_2} + {v_4}{h_3}} \right)\]

\textbf{Question 1-2}. \textbf{Derive a learning algorithm for v2, w5, using LMS (Least Mean Square)
method, i.e., by minimizing the output error e = t - y, where t is the target output.}
\\

\\
\textbf{Solution 1-2:}
\\
Given that $f(x)=tanh(x)$ and $g(x)=x$, we have $f'$ and $g'$:
\[f' = 1 - {f^2}\]
\[g' = 1\]

By observation from the diagram, $v_2$ is a directed Edge from Vertex $h_1$ to Vertex $y$; while, $w_5$ is a directed Edge from Vertex $x_2$ to $h_3$.

From the lecture, we can proofed that the LMS Error strategy leads to a uniform learning equation for both the hidden layer and the output layer:
\[\Delta Edge = \eta  \times \underbrace {{{\left( {Activation Function} \right)}^\prime }}_{{\rm{Activation function @ Dist}}} \times \underbrace {\left( {error} \right)}_{{\rm{error @ Dist}}} \times \underbrace {\left( {input} \right)}_{{\rm{value @ Source}}}\]

This is to say
\begin{equation}
     \boxed{\Delta {v_2} = \eta g'{e_y}{h_1} = \Delta {w_3} = \eta \left( {t - y} \right){h_1}}\label{eq1-2-1}
\end{equation}
\begin{equation}
    \boxed{\Delta {w_5} = \eta f'{e_{h3}}{x_2} = \Delta {w_3} = \eta \left( {1 - {f^2}} \right){e_{h3}}{x_2}}\label{eq1-2-2}
\end{equation}


Where:
\[{e_{h3}} = g' \times e \times {v_4} = g'\left( {t - y} \right){v_4} = \left( {t - y} \right){v_4}\]
\noindent{\color{red} \rule{\linewidth}{0.25mm}}
% ==========================================================================================
To get equation \eqref{eq1-2-1}, and equation \eqref{eq1-2-2} in detail:

The LMS Error $E$ is defined as:
\[E = \frac{1}{2}{\left( {t - y} \right)^2}\]
It is trivial to get $\Delta{v_2}$
\[\Delta {v_2} =  - \eta \frac{{\partial E}}{{\partial {v_2}}} =  - \eta \left( {t - y} \right)\left( {\frac{{\partial \left( {t - y} \right)}}{{\partial {v_2}}}} \right) = \eta \left( {t - y} \right)\left( {\frac{{\partial y}}{{\partial {v_2}}}} \right) = \eta \left( {t - y} \right)\left( {{h_1}g'} \right) = \eta \left( {t - y} \right){h_1} \qed\]
It is tricky to calculate $\Delta{w_5}$
\[\Delta {w_5} =  - \eta \frac{{\partial E}}{{\partial {w_5}}} =  - \eta \left( {t - y} \right)\left( {\frac{{\partial \left( {t - y} \right)}}{{\partial {w_5}}}} \right) = \eta \left( {t - y} \right)\left( {\frac{{\partial y}}{{\partial {w_5}}}} \right)\]
Re-write $y$ so that $y$ explicitly contains $w_5$:
\[y = g\left( {{v_1}x{ & _1} + {v_2}{h_1} + {v_3}{h_2} + {v_4}{h_3}} \right) = g\left[ {\left( {{v_1}x{ & _1} + {v_2}{h_1} + {v_3}{h_2}} \right) + {v_4}f\left( {{w_3}{x_1} + {w_5}{x_2} + {w_7}{x_3} + {\theta _{h3}}} \right)} \right]\]
\[\frac{{\partial y}}{{\partial {w_5}}} = g'\frac{    {\partial \left[ { \left( {{v_1}x{ & _1} + {v_2}{h_1} + {v_3}{h_2}} \right) + {v_4}f\left( {{w_3}{x_1} + {w_5}{x_2} + {w_7}{x_3} + {\theta _{h3}}} \right)} \right]      }     }   {{\partial {w_5}}} = g'{v_4}f'{x_2} = {v_4}f'{x_2}\]

So, we can get:
\[\Delta {w_5} = \eta \left( {t - y} \right)\left( {\frac{{\partial y}}{{\partial {w_5}}}} \right) = \eta \underbrace {\left( {t - y} \right){v_4}}_{{e_{h3}}}f'{x_2} = \eta {e_{h3}}f'{x_2} = \eta \left( {1 - {f^2}} \right){e_{h3}}{x_2}\qed\]




\textbf{Question 2-1. Design a single-neuron perceptron to solve this problem.}

\begin{center}
\begin{tikzpicture}
\Vertex[x=0,RGB,color={190,174,212},label=\(x_0\)]{A} 
\Vertex[x=2,label=\(x_1\)]{B}
\Vertex[x=4,label=\(x_2\)]{C}


\Vertex[x=2,y=-2,label=Neuron,size = 1]{E}
\Vertex[x=2,y=-4, label=y]{F}

\Edge[bend=-10,Direct=true,label=$\theta$](A)(E)
\Edge[bend=-5,Direct=true,label=w1](B)(E)
\Edge[bend=5,Direct=true,label=w2](C)(E)

\Edge[bend=0,Direct=true](E)(F)
\end{tikzpicture}
\end{center}

%=============== two minipage===============================
\noindent
\begin{minipage}{.5\textwidth}
  \begin{center}
\begin{tabular}{|l|l|l|l|l}
\hline
            & \textbf{X1} & \textbf{X2} & \textbf{t} \\ \hline
\textbf{P1} & -1          & 1         & 1         \\ \cline{1-4}
\textbf{P2} & -1           & -1           & 1          \\ \cline{1-4}
\textbf{P3} & 0           & 0           & 0          \\\cline{1-4}
\textbf{P4} & 1           & 0           & 0          \\ \hline
\end{tabular}
\end{center}
\end{minipage}% This must go next to `\end{minipage}`
\begin{minipage}{.5\textwidth}
  \begin{center}
\begin{tikzpicture}
\begin{axis}[
    axis lines=middle,
    xmin=-1.5, xmax=1.5,
    ymin=-1.5, ymax=1.5,
    xlabel={$x1$},
    ylabel={$x2$}],
    
    ]
\fill[green!40!white, opacity=0.3] (100,0) rectangle (300,400);
\addplot [smooth,blue,name path=A,no marks]coordinates{(-0.5,2)(-0.5,0)(-0.5,-2)}; % actual curve
% \addplot [draw=none,name path=B, no marks] {-2};     % “fictional” curve
% \addplot [green!40,opacity=0.5] fill between[of=A and B,soft clip={domain=-4:4}]; % filling

\addplot[color=red,only marks,] coordinates {(-1,1)(-1,-1)};
\addplot[color=red, only marks, mark=triangle*,mark size = 3pt]coordinates{(0,0)};
\addplot[color=red, only marks, mark=triangle*,mark size=3pt]coordinates{(1,0)};
\addplot[black](0.3,0.5) circle (0pt) node[anchor=west] {\large t=0 region};

\addplot[black](-1,1) circle (4pt) node[anchor=south] {P1};
\addplot[black](-1,-1) circle (4pt) node[anchor=south] {P2};
\addplot[black](0,0) circle (4pt) node[anchor=west] {P3};
\addplot[black](1,0) circle (4pt) node[anchor=west] {P4};
\end{axis}
\end{tikzpicture}
\end{center}
\end{minipage}

%===============end of  two minipage===============================

\textbf{Solution2-1:}

For hardlimitor activation function $y=h.l.(\mathbb{W}\mathbb{X}+\theta)$, the decision boundary is determined by:
\begin{equation}
    \mathbb{W}\mathbb{X} + \theta  = 0\label{eq2-1-1}
\end{equation}

By observation, we are going to choose the line $x_1=-0.5$ as our boundary. The weight vector that is orthogonal to our decision boundary is:
\[\mathbb{W} = \left[ {\begin{array}{*{20}{c}}
{{w_1}}&{{w_2}}
\end{array}} \right] = \left[ {\begin{array}{*{20}{c}}
-1&0
\end{array}} \right]\]


We know this $x_1=-0.5$ line passing point $(-0.5, 1)$, we can find $\theta$ by
\[\left[ {\begin{array}{*{20}{c}}
-1&0
\end{array}} \right]\left[ {\begin{array}{*{20}{c}}
{ - 0.5}\\
1
\end{array}} \right] + \theta  =  0.5  + \theta  = 0 \Rightarrow \theta  = -0.5\]
So we can write our model as:
\begin{equation}
    \boxed{y = h.l.\left( {\left[ {\begin{array}{*{20}{c}}
-1&0
\end{array}} \right]\left[ {\begin{array}{*{20}{c}}
{{x_1}}\\
{{x_2}}
\end{array}} \right] - 0.5} \right)}
\end{equation}

\textbf{Question-2-2: Test your solution with all four input vectors.}

\textbf{Solution-2-2:}

Use all 4 input points to test our model:
\begin{enumerate}
    \item Test for P1:
    \[{y_1} = h.l.\left( {\left[ {\begin{array}{*{20}{c}}
-1&0
\end{array}} \right]\left[ {\begin{array}{*{20}{c}}
{ - 1}\\
1
\end{array}} \right] -0.5} \right) = h.l.\left( { 1 + 1} \right) = 1 \qquad \text{\rlap{$\checkmark$}}\square\] 

\item Test for P2:
\[{y_2} = h.l.\left( {\left[ {\begin{array}{*{20}{c}}
-1&0
\end{array}} \right]\left[ {\begin{array}{*{20}{c}}
{ - 1}\\
{ - 1}
\end{array}} \right] -0.5} \right) = h.l.\left( {  1 + 1} \right) = 1\qquad \text{\rlap{$\checkmark$}}\square\] 


\item Test for P3:
\[{y_3} = h.l.\left( {\left[ {\begin{array}{*{20}{c}}
{ - 1}&0
\end{array}} \right]\left[ {\begin{array}{*{20}{c}}
0\\
0
\end{array}} \right]  - 0.5} \right) = h.l.\left( { 0 - 0.5} \right) = 0\qquad {\rlap{$\checkmark$}}\square\]

\item Test for P4:
\[{y_4} = h.l.\left( {\left[ {\begin{array}{*{20}{c}}
1&0
\end{array}} \right]\left[ {\begin{array}{*{20}{c}}
0\\
0
\end{array}} \right] - 0.5} \right) = h.l.\left( { 0 - 0.5} \right) = 0\qquad {\rlap{$\checkmark$}}\square\]
\end{enumerate}

\textbf{Question-2-3: Classify the following input vectors with your solution. You can either perform
the calculations manually or with Matlab.}


\textbf{Solution-2-3:}

%=============== two minipage===============================
\noindent
\begin{minipage}{.5\textwidth}
  \begin{center}
\begin{tabular}{|l|l|l|l|l}
\hline
            & \textbf{X1} & \textbf{X2} & \textbf{Y} \\ \hline
\textbf{P5} & -2          & 0         &  \textcolor{red}{1}         \\ \cline{1-4}
\textbf{P6} & 1           & 1           & \textcolor{red}{0}          \\ \cline{1-4}
\textbf{P7} & 0           & 1           & \textcolor{red}{0}          \\\cline{1-4}
\textbf{P8} & -1          & 2           & \textcolor{red}{1}          \\ \hline
\end{tabular}
\end{center}
\end{minipage}% This must go next to `\end{minipage}`
\begin{minipage}{.5\textwidth}
  \begin{center}
\begin{tikzpicture}
\begin{axis}[
    axis lines=middle,
    xmin=-2.5, xmax=2.5,
    ymin=-2.5, ymax=2.5,
    xlabel={$x1$},
    ylabel={$x2$}],
    
    ]
\fill[green!40!white, opacity=0.3] (200,0) rectangle (600,600);
\addplot [smooth,blue,name path=A,no marks]coordinates{(-0.5,3)(-0.5,0)(-0.5,-3)}; % actual curve
% \addplot [draw=none,name path=B, no marks] {-2};     % “fictional” curve
% \addplot [green!40,opacity=0.5] fill between[of=A and B,soft clip={domain=-4:4}]; % filling

\addplot[color=red,only marks,] coordinates {(-2,0)(-1,2)};
\addplot[color=red, only marks, mark=triangle*,mark size = 3pt]coordinates{(0,1)};
\addplot[color=red, only marks, mark=triangle*,mark size=3pt]coordinates{(1,1)};
\addplot[black](0.3,0.5) circle (0pt) node[anchor=west] {\large y=0 region};

\addplot[black](-2,0) circle (4pt) node[anchor=south] {P5};
\addplot[black](1,1) circle (4pt) node[anchor=south] {P6};
\addplot[black](0,1) circle (4pt) node[anchor=west] {P7};
\addplot[black](-1,2) circle (4pt) node[anchor=west] {P8};
\end{axis}
\end{tikzpicture}
\end{center}
\end{minipage}

%===============end of  two minipage===============================


\begin{lstlisting}
% Learning Model
w=[-1,0]
theta=-0.5

% Training DataSet P1-P4 for question2-1 to question2-2 
pointset1=[-1,-1,0,1;1,-1,0,0]
result1 = hardlim(w*pointset1+theta)

% DataSet P5--P8 for question2-3
pointset2=[-2,1,0,-1;0,1,1,-2]
result2 = hardlim(w*pointset2+theta)
% result2 =
%      1     0     0     1
\end{lstlisting}

\textbf{Question-2-4: Which of the input vectors in Part (3) is always classified the same way,
regardless of the solution values of the weigh $\mathbb{W}$ and the threshold $\theta$? Which may
vary depending on the solution? Why?}

\begin{center}
\begin{tikzpicture}
\begin{axis}[
    axis lines=middle,
    xmin=-2.5, xmax=2.5,
    ymin=-2.5, ymax=2.5,
    xlabel={$x1$},
    ylabel={$x2$}],
    
    ]
\fill[green!40!white, opacity=0.3] (200,0) rectangle (600,600);
\addplot [smooth,blue,name path=A,no marks]coordinates{(-0.5,3)(-0.5,0)(-0.5,-3)}; % actual curve
% \addplot [draw=none,name path=B, no marks] {-2};     % “fictional” curve
% \addplot [green!40,opacity=0.5] fill between[of=A and B,soft clip={domain=-4:4}]; % filling

\addplot[color=red,only marks,] coordinates {(-2,0)(-1,2)};
\addplot[color=red, only marks, mark=triangle*,mark size = 3pt]coordinates{(0,1)};
\addplot[color=red, only marks, mark=triangle*,mark size=3pt]coordinates{(1,1)};
\addplot[black](0.3,1.8) circle (0pt) node[anchor=west] {\large y=0 region};

\addplot[black](-2,0) circle (4pt) node[anchor=south] {P5};
\addplot[black](1,1) circle (4pt) node[anchor=south] {P6};
\addplot[black](0,1) circle (4pt) node[anchor=west] {P7};
\addplot[black](-1,2) circle (4pt) node[anchor=west] {P8};

\addplot[color=gray,only marks,] coordinates {(-1,1)(-1,-1)};
\addplot[color=blue, only marks, mark=triangle*,mark size = 3pt]coordinates{(0,0)};
\addplot[color=blue, only marks, mark=triangle*,mark size=3pt]coordinates{(1,0)};

\addplot[black](-1,1) circle (4pt) node[anchor=south] {P1};
\addplot[black](-1,-1) circle (4pt) node[anchor=south] {P2};
\addplot[black](0,0) circle (4pt) node[anchor=west] {P3};
\addplot[black](1,0) circle (4pt) node[anchor=west] {P4};

\addplot [smooth,blue,name path=X,no marks,style=very thick]coordinates{(-1,-1)(-1,0)(-1,1)}; % actual curve
\addplot [smooth,blue,name path=X,no marks,style=very thick]coordinates{(0,0)(0.5,0)(1,0)}; % actual curve
\end{axis}
\end{tikzpicture}
\end{center}

\textbf{Solution:}
We can see from the plot below, no decision boundary shall across the \textcolor{orange}{two thick orange lines} that respectively connect \textcolor{gray}{Class-1 (Point-1 and Point-2)} and \textcolor{blue}{Class-2 (Point-3 and Point-4)}.
\begin{itemize}
    \item Any data points inside of the gray zone (open boundary) are \textbf{not} always classified the same way.
    \item Any data points inside of the white zone (close boundary) are always classified the same way
\end{itemize}
\begin{equation*}
    \boxed{\rm{\quad Point\,7 \, and\, Point\,8\,may\,vary\,depending\,on\,the\,decision\,boundary\quad}}
\end{equation}
\begin{equation*}
    \boxed{\rm{\quad Point\,5 \, and\, Point\,6\,can\,always\,be\,classified\,the\,same\,way\quad\quad\quad\quad\quad}}
\end{equation}

% I do not have a good mathematical proof, but my intuition tells me that the \textit{boundary of the decision boundary} could be determined by the following theorem:


% \begin{theorem}[Yaowen's Last Theorem]
% \label{pythagorean}
% On a 2-D Plane, two non-intersecting \textbf{Convex polygon}, A and B, can always be separated by \textbf{exactly} two lines that are intersecting with both polygon A and polygon B. These two lines must passing from one vertex from polygon A and one vertex from polygon B, not touching any edges of any polygons. These two lines could uniquely determiner the \textbf{boundary of decision boundary} that will separate these two polygons on the 2-D plane. 
% \end{theorem}


\begin{center}
\begin{tikzpicture}
\begin{axis}[
    axis lines=middle,
    xmin=-2.5, xmax=2.5,
    ymin=-2.5, ymax=2.5,
    xlabel={$x1$},
    ylabel={$x2$}],
    
    ]
% \fill[green!40!white, opacity=0.3] (200,0) rectangle (600,600);
\coordinate (l1) at (0,0);
\coordinate (l2) at (250,250);
\coordinate (l3) at (500,0);
\coordinate (h1) at (0,500);
\coordinate (h3) at (500,500);
\coordinate (m1) at (150,150);
\coordinate (m3) at (150,350);

\filldraw[draw=black, fill=gray!20,opacity=0.3] (l1) -- (l2) -- (l3)--(l1);
\filldraw[draw=black, fill=gray!20,opacity=0.3] (h1) -- (l2) -- (h3)--(h1);
\filldraw[draw=black, fill=gray!20,opacity=0.3] (m1) -- (l2) -- (m3)--(m1);

\addplot[color=red,only marks,] coordinates {(-2,0)(-1,2)};
\addplot[color=red, only marks, mark=triangle*,mark size = 3pt]coordinates{(0,1)};
\addplot[color=red, only marks, mark=triangle*,mark size=3pt]coordinates{(1,1)};
% \addplot[black](0.3,1.8) circle (0pt) node[anchor=west] {\large y=0 region};

\addplot[black](-2,0) circle (4pt) node[anchor=south] {P5};
\addplot[black](1,1) circle (4pt) node[anchor=south] {P6};
\addplot[black](0,1) circle (4pt) node[anchor=west] {P7};
\addplot[black](-1,2) circle (4pt) node[anchor=west] {P8};

\addplot[color=gray,only marks,] coordinates {(-1,1)(-1,-1)};
\addplot[color=blue, only marks, mark=triangle*,mark size = 3pt]coordinates{(0,0)};
\addplot[color=blue, only marks, mark=triangle*,mark size=3pt]coordinates{(1,0)};

\addplot[black](-1,1) circle (4pt) node[anchor=south] {P1};
\addplot[black](-1,-1) circle (4pt) node[anchor=south] {P2};
\addplot[black](0,0) circle (4pt) node[anchor=west] {P3};
\addplot[black](1,0) circle (4pt) node[anchor=west] {P4};

% The lines
\addplot [smooth,orange,name path=X,no marks,style=very thick]coordinates{(-1,-1)(-1,0)(-1,1)}; % actual curve
\addplot [smooth,orange,name path=X,no marks,style=very thick]coordinates{(0,0)(0.5,0)(1,0)}; % actual curve
\addplot [smooth,blue,name path=AA,no marks,style= dashed]coordinates{(-2.5,2.5)(-1,1)(0,0)(2.5,-2.5)}; % actual curve
\addplot [smooth,blue,name path=AB,no marks,style= dashed]coordinates{(-2.5,-2.5)(0,0)(2.5,2.5)}; % actual curve
\end{axis}
\end{tikzpicture}
\tikzset{
    circ/.style={circle, fill=black, inner sep=2pt, node contents={}}
}

\end{document}