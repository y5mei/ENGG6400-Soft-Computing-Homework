1. ¿ Cómo se relacionan los renglones de $EA$ con los de $A$ en los siguientes casos? ( suponga que $A$ es una matriz cualesquiera)\\

E=$\begin{pmatrix}
 1& 0& 0 \\
 0& 2& 0 \\ 
 4& 0& 1 
\end{pmatrix}$\\
\begin{itemize}
\item La primera fila de EA y A son iguales 
\item La segunda fila de EA es dos veces la segunda fila de A
\item La tercera fila es EA es cuatro veces la primera fila de A más la tercera fila de A
\end{itemize}
E=$\begin{pmatrix}
 1& 1& 1 \\
 0& 0& 0 
\end{pmatrix}$\\
\begin{itemize}
\item La componente n de la primera fila de EA es la suma de los componentes de la columna n de A
\item La segunda fila es una fila de ceros 
\end{itemize}
E=$\begin{pmatrix}
 0& 0& 1 \\
 0& 1& 0 \\
 1& 0& 0
\end{pmatrix}$\\
\begin{itemize}
\item La primera fila de A y la tercera fila de A se intercambian 
\end{itemize}
2. Determinar si la proposición es falsa o verdadera (justificar) $A,B$ son matrices cuadradas de orden n.\\
\begin{itemize}
\item det(A+B)= det(A) + det(B) (falsa)\\

Contra ejemplo: \\
Si A es igual a B, A es una matriz cuadrada de orden cuatro y el determinante de A es 4:\\
det(A+B)= 64 y det(A)+det(B)=8 
\item Si $A^{k} = 0_{nxn}$ para algún k entero positivo entonces A es singular (verdadera)\\
Si $A^{k} = 0_{nxn}$, es decir es nilpotente, el determinante debe ser cero razón por la cual la matriz es singular
\item Si det(A)=-2, entonces el sistema $AX=0$ tiene solo la solución trivial (verdadera)\\
A es invertible por ser su determinante diferente de 0 por lo cual $X= A^{-1} 0$
\item Si A es idempotente entonces det(A)=0 (falsa)
La matriz identidad es idempotente y su determinante es 1
\end{itemize}
3. Tres vectores $\vec{u},\vec{v},\vec{w}$ de $\Re^{3}$ satisfacen que:\\

$\left| \left| \vec{u} \right|\right |$=$\left | \left |\vec{w} \right |\right |$=5; $\left| \left| \vec{v} \right|\right |$ = 1; $\left| \left| \vec{u}-\vec{v}+\vec{w} \right|\right|=\left| \left| \vec{u}+\vec{v}+\vec{w} \right|\right|$\\

Si el angulo que forman $\vec{u}$ y $\vec{v}$ es $\frac{\Pi}{8}$, hallar el que forman $\vec{v}$ y $\vec{w}$\\

$\left(\vec{u}-\vec{v}+\vec{w}\right) \cdot \left(\vec{u}-\vec{v}+\vec{w}\right)$=$\left(\vec{u}+ \vec{v}+\vec{w}\right) \cdot \left(\vec{u}+\vec{v}+\vec{w}\right)$\\

$\vec{u} \cdot \vec{u} + \vec{v} \cdot \vec{v} + \vec{w} \cdot \vec{w} - 2 \vec{u} \cdot \vec{v} + 2 \vec{u} \cdot \vec{w} - 2 \vec{v} \cdot \vec{w}=\vec{u} \cdot \vec{u} + \vec{v} \cdot \vec{v} + \vec{w} \cdot \vec{w} + 2 \vec{u} \cdot \vec{v} + 2 \vec{u} \cdot \vec{w} + 2 \vec{v} \cdot \vec{w}$\\

$- 2 \vec{u} \cdot \vec{v} - 2 \vec{v} \cdot \vec{w}= 2 \vec{u} \cdot \vec{v} + 2 \vec{v} \cdot \vec{w}$\\

$ - \vec{v} \cdot \vec{w}= \vec{u} \cdot \vec{v}$\\

$cos(\theta)=\frac{\vec{v} \cdot \vec{w}}{\left| \left| \vec{v} \right|\right |\left| \left| \vec{w} \right|\right |}$\\

$cos(\theta) =\frac {- \vec{u} \cdot \vec{v}}{5}$\\

$\theta = -\frac{\Pi}{8}$\\

4. Escriba una ecuación lineal de n variables , diga que representa:\\

$a_{1}x_{1}+a_{2}x_{2}+a_{3}x_{3}+...+a{n}x_{n}=b$ \\

Representa un plano en el espacio $\Re^{n}$\\

5. Demuestre que el conjunto $\vec{B}$ es una base de $\vec{V}$ y encuentre el vector de coordenadas de $\vec{u}$ respecto a la base $\vec{B} (\left [ \vec{u} \right ]_{\vec{B}})$\\

\begin{itemize}
\item $\vec{B}= \left\lbrace\ (3,2,2), (-1,2,1),(0,1,0)\right\rbrace\ $, $\vec{V}= \Re^{3}$, $\vec{u}=(5,3,1)$\\

$\begin{vmatrix}
 3& -1 & 0\\ 
 2& 2 & 1\\ 
 2& 1 & 0
\end{vmatrix}$ = 5, por lo tanto si es base de $\Re^{3}$\\


$(\left [ \vec{u} \right ]_{\vec{B}})=\frac{1}{5}\begin{pmatrix}
1& 0 & 1\\ 
-2& 0 & 3\\ 
2& 5 & -8
\end{pmatrix}$ $\begin{bmatrix}
5\\ 
3\\ 
1
\end{bmatrix}$ = $\frac{1}{5}\begin{bmatrix}
6\\ 
-7\\ 
17
\end{bmatrix}$
\item $\vec{B}= \left\lbrace\ x^{2}+x; x-1;x+1\right\rbrace\ $, $\vec{V}= P_{2}$, $\vec{u}=3x^{2}-x+2$\\

$\begin{vmatrix}
 1& 0 & 0\\ 
 1& 1 & 1\\ 
 0& -1 & 1
\end{vmatrix}$ = 2, por lo tanto si es base de $P_{2}$\\


$(\left [ \vec{u} \right ]_{\vec{B}})=\frac{1}{2}\begin{pmatrix}
2& 0 & 0\\ 
-1& 1 & -1\\ 
-1& 1 & 1
\end{pmatrix}$ $\begin{bmatrix}
3\\ 
-1\\ 
2
\end{bmatrix}$ = $\begin{bmatrix}
3\\ 
-3\\ 
-1
\end{bmatrix}$=$3x^{2}-3x-1$
\item $\vec{B}= \left\lbrace\ \begin{bmatrix}
 1& 1\\ 
 0& 0
\end{bmatrix},\begin{bmatrix}
 0& 0\\ 
 1& 1
\end{bmatrix},\begin{bmatrix}
 1& 0\\ 
 0& 1
\end{bmatrix},\begin{bmatrix}
 0& 1\\ 
 1& 1
\end{bmatrix}\right\rbrace\ $

$\begin{vmatrix}
 1& 1 & 0&0\\ 
 0& 0 & 1&1\\ 
 1& 0 & 1&0\\
 0& 1 & 1&1
\end{vmatrix}$ = -1, por lo tanto si es base. 
\end{itemize}

6.Demostrar que si A y B son semejantes:\\

\begin{itemize}
\item det(A)=det(B)\\
PA=BP\\
det(PA)=det(BP)
det(P)det(A)=det(B)det(P)
det(A)=det(B)
\item tr(PA)=tr(BP)\\
tr(A)tr(P)=tr(P)tr(B)\\
PA=PB
\item A y B tienen el mismo polinomio característico\\
det(P(A-$\lambda$I ))=det((B-$\lambda$I)P)\\
det(A-$\lambda$I)=det(B-$\lambda$I)\\
Como tienen el mismo polinomio característico tienen los mismos valores propios 
\item $PA^{n}=B^{n}P$\\
$det(PA^{n})=det(B^{n}P)$\\
$det(A^{n})=det(B^{n})$
\item $PA^{-1}=B^{-1}P$\\
$AP^{-1}=P^{-1}B$\\
$A=P^{-1}BP$
\end{itemize}
